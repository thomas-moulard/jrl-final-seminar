% -*- mode: latex-mode -*-

\usepackage{etex}

% http://ctan.org/pkg/media9
\usepackage{media9}

\usepackage[utf8]{inputenc} % hyperref broken with utf8x
\usepackage[C40,T1]{fontenc}

\usepackage{lmodern}
\usepackage{graphicx}
\usepackage{array}
\usepackage{multirow}
\usepackage{overpic}
\usepackage{amsmath}

\usepackage{algorithmic, algorithm}

\usepackage{tikz, pgfplots}
\usepackage{tikz-qtree}
\usepackage{pgf-umlcd}

\newcommand{\umltextcolor}{ThoughfulBrick}
\newcommand{\umldrawcolor}{Thoughtless}
\newcommand{\umlfillcolor}{ThoughtBySome}


\usetikzlibrary{calc,trees,positioning,arrows,chains,shapes.geometric,%
    decorations.pathreplacing,decorations.pathmorphing,shapes,%
    matrix,shapes.symbols}

\tikzstyle{every picture}+=[remember picture]
\tikzstyle{na} = [baseline=-.5ex]

\tikzset{
>=stealth',
  punktchain/.style={
    rectangle,
    rounded corners,
    % fill=black!10,
    draw=black, very thick,
    text width=10em,
    minimum height=3em,
    text centered,
    on chain},
  line/.style={draw, thick, <-},
  element/.style={
    tape,
    top color=white,
    bottom color=blue!50!black!60!,
    minimum width=8em,
    draw=blue!40!black!90, very thick,
    text width=10em,
    minimum height=3.5em,
    text centered,
    on chain},
  every join/.style={->, thick,shorten >=1pt},
  decoration={brace},
  tuborg/.style={decorate},
  tubnode/.style={midway, right=2pt},
  template parameter/.style={
        append after command={
            node [draw, densely dashed, umlcolor, font=\ttfamily]
                at (\tikzlastnode.north east)
                {#1}
        }
    }
}

% Remap the unicode indivisible space into a LateX tilde.
\DeclareUnicodeCharacter{00A0}{~}

% Customizme Beamer.
\usetheme{jrl}

% Set title, author, date.
\title[RobOptim]{RobOptim: a Toolbox for Numerical Optimization in Robotics}
\author[\emph{T.\ Moulard}]{Thomas Moulard}

\institute{CNRS-AIST JRL UMI3218/CRT}

\date{Friday 27 June 2014}

% Setup pdf meta-data.
\hypersetup{
  pdfauthor = {Thomas Moulard},
  pdftitle = {RobOptim},
  pdfsubject = {RobOptim},
  pdfkeywords = {numerical optimization, robotics},
  pdfcreator = {Thomas Moulard},
  pdfproducer = {Thomas Moulard}
}

\begin{document}

\begin{frame}[plain]
  \titlepage
\end{frame}

\begin{frame}
  \tableofcontents
\end{frame}

%%%%%%%%%%%%%%%%%%%%%%%%%%%%%%%%%%%%%%%%%%%%%%%%%%%%%%%%%%%%%%%%%%%%%%%%%%%%%%%%
%%%%%%%%%%%%%%%%%%%%%%%%%%%%%%%%%%%%%%%%%%%%%%%%%%%%%%%%%%%%%%%%%%%%%%%%%%%%%%%%
%%%%%%%%%%%%%%%%%%%%%%%%%%%%%%%%%%%%%%%%%%%%%%%%%%%%%%%%%%%%%%%%%%%%%%%%%%%%%%%%
\section{RobOptim}
\begin{frame}
   \vfill
   \begin{center}
     \Large I. RobOptim
   \end{center}
   \vfill
\end{frame}

\begin{frame}
  \frametitle{Optimisation}

  \begin{equation*}
    \min_{\mathbf{x} \in \mathbb{R}^n} f(\mathbf{x})%
    \text{ under the constraints } \mathbf{x} \in \mathbf{X}
  \end{equation*}
  
  \begin{equation*}
    \mathbf{X} \equiv \left\{
    \begin{array}{l l}
      c_i (x) = 0    & \quad i \in \mathcal{E} \\
      c_j (x) \leq 0 & \quad j \in \mathcal{I} \\
    \end{array} \right.
  \end{equation*}
\end{frame}


\begin{frame}
  
    \frametitle{Motivation}

    \begin{center}
      \begin{description}
        \item[Model]~\\
          A lot of solvers, no interface.
        \item[Generic]~\\
          Separate the solver from the problem.
        \item[Polyvalent]~\\
          Versatile tool for robotics.
      \end{description}
    \end{center}
    \vspace{1cm}
    
    What is the \alert{computer representation} of an optimization problem?
  
\end{frame}


\begin{frame}
  \frametitle{Functions (1)}

    \begin{center}
      \footnotesize
    \begin{tikzpicture}[auto]

      \begin{interface}[text width=7cm]{Function}{0,0}
        \operation{+ evaluation}
      \end{interface}

      \begin{interface}[text width=7cm]{Differentiable Function}{0,-3}
        \inherit{Function}
        \operation{+ jacobian}
      \end{interface}

      \begin{interface}[text width=7cm]{Twice Differentiable Function}{0,-6}
        \operation{+ hessian}
      \end{interface}

      \path[draw=\umldrawcolor, open triangle 45-, ultra thick]
      (Function.south) -- (Differentiable Function.north);
      \path[draw=\umldrawcolor, open triangle 45-, ultra thick]
      (Differentiable Function.south) -- (Twice Differentiable Function.north);
    \end{tikzpicture}
    \end{center}
\end{frame}


\begin{frame}
  \frametitle{Functions (2)}

    \begin{center}
      \footnotesize
    \begin{tikzpicture}[auto]

      \begin{interface}[text width=7cm]{Twice Differentiable Function}{0,0}
        \operation{+ hessian (vector, int): vector}
      \end{interface}

      \begin{interface}[template parameter=n,text width=7cm]{%
          ``n'' Differentiable Function fois}{0,-3}
        \operation{+ derivative}
      \end{interface}

      \path[draw=\umldrawcolor, open triangle 45-, ultra thick]
      (Twice Differentiable Function.south) node[below right] {$n=3$}
      -- (``n'' Differentiable Function fois.north);

      \path[draw=\umldrawcolor, open triangle 45-, ultra thick]
      (``n'' Differentiable Function fois.south)
      -- ++(0, -1)
      -- ++(-1, 0) node[below] {$n > 3$}
      -- ++(0, 1);

    \end{tikzpicture}
    \end{center}
  
\end{frame}

\begin{frame}
  \frametitle{Problem}

    \begin{center}
      \footnotesize
    \begin{tikzpicture}[auto]

      \begin{interface}[template parameter={F,$C_1$,$\dotsc$,$C_n$},%
          text width=7cm]{Problem}{0,0}
        \attribute{+ starting point: vector}
        \attribute{+ bounds: (double, double)[*]}
        \attribute{+ cost: \texttt{$F$}}
        \attribute{+ constraints: (\texttt{$C_i$}, int, int)[*]}
      \end{interface}
    \end{tikzpicture}

    \begin{equation*}
      \begin{split}
        F <: \text{Function}\\
        \forall i, C_i <: \text{Function}
      \end{split}
    \end{equation*}
    \end{center}
  
\end{frame}

\begin{frame}
  \frametitle{Solver}
  \begin{center}
    \footnotesize
    \begin{tikzpicture}[%
        y=.1\paperheight,
        auto]
      
      \node[na] at (5,-1) (P) {P};
      
      \begin{interface}[%
          template parameter={Problem<$F$,$C_1$,$\dotsc$,$C_n$>},
          text width=7cm]{Solver}{0,0}
        %\inherit{P}
        \attribute{+ problem: $\text{Problem}<F, C_1, \dotsc, C_n>$}
        \operation{+ solve (): vector}
      \end{interface}

      \composition{Solver}{}{1}{P}
    \end{tikzpicture}
    
    \begin{equation*}
      (F, C_1, \dotsc, C_n) <: Function^{n+1}
    \end{equation*}
  \end{center}  
\end{frame}

\begin{frame}  
    \frametitle{Finite Differences}

    \begin{center}
      \footnotesize
      \begin{tikzpicture}[%
          y=.1\paperheight,
          auto]

      \node[na] at (0,0) (F) {F};

        \begin{interface}[template parameter=F,%
            text width=7cm]{FiniteDifferences}{0,-2}
          %\inherit{F}
          \attribute{- function: F}
          \operation{+ jacobian (vector): vector}
        \end{interface}

      \path[draw=\umldrawcolor, open triangle 45-, ultra thick]
      (F.south) -- (FiniteDifferences.north);
      \end{tikzpicture}
    \end{center}
    \bigskip
    Design Pattern \alert{Decorator}.
  
\end{frame}


\begin{frame}
  \frametitle{Strong Typing}

    \begin{center}
      \begin{description}
        \item[Correctness]~\\
          All compilable problems are valid.

        \item[Genericity]~\\
          A solver can handle problems whose complexity is ``lower''
          or ``equal'' to what it can handle.
          
        \item[Efficient]~\\
          A solver can handle problems whose complexity is ``greater''
          than what it can handle but it must be done explicitly.
      \end{description}
    \end{center}
  
\end{frame}


\begin{frame}
  \frametitle{Showcase (I)}

  \begin{center}
    \includegraphics[height=.33\paperheight]{%
      assets/chap1-roboptim/straight-line-obstacle.png}
    \includegraphics[height=.33\paperheight]{%
      assets/slides/agent-067.jpg}\par
    \includegraphics[height=.33\paperheight]{%
      assets/slides/multi-014.jpg}
    \includegraphics[height=.33\paperheight]{%
      assets/slides/multi-015.jpg}

    Figures 2, 3, 4: K. Bouyarmane et al.
  \end{center}
\end{frame}

\begin{frame}
  \frametitle{Showcase (II)}

  \begin{center}
    \includegraphics[height=.25\paperheight]{%
      assets/capsule.png}\\
    A. El Khoury et al.\\
    \bigskip
    \includegraphics[height=.25\paperheight]{%
      assets/ellipses.png}\par
    S. Brossette et al.
  \end{center}
\end{frame}


\begin{frame}
  \frametitle{RobOptim Architecture}

  \begin{center}
    \tikzstyle{roboptim} = [align=center,
      rectangle,
      minimum size=6mm,
      minimum height=15mm,
      color=Thoughtless,
      fill=ThoughtBySome,
      line width=1pt,
      inner sep=0pt,
      outer sep=0pt,
      text width=70pt
    ]

    \begin{tabular}{cccl}
      \tikz \node[roboptim] (cfsqp) {CFSQP}; &
      \tikz \node[roboptim] (ipopt) {IPOPT}; &
      \tikz \node[roboptim] (cminpack) {CMinPack}; &
      \parbox[l][1.5cm][l]{2cm}{%
        \vspace{-.6cm}\rotatebox{-90}{\footnotesize \textbf{Solver}}}\\
      \multicolumn{3}{c}{%
        \tikz \node[roboptim,
          text width=235pt] (core) {RobOptim Core};
      } &
      \parbox[l][1.5cm][l]{2cm}{%
        \vspace{-.6cm}\rotatebox{-90}{\footnotesize \textbf{Interface}}}\\
      \tikz \node[roboptim] (traj) {RobOptim Trajectory}; &
      \tikz \node[roboptim] (post) {RobOptim Posture}; &
      \tikz \node[roboptim] (others2) {\ldots}; &
      \parbox[l][1.5cm][l]{2cm}{%
        \vspace{-.6cm}\rotatebox{-90}{%
          \footnotesize \textbf{Application}}}\\
    \end{tabular}
  \end{center}
\end{frame}



%%%%%%%%%%%%%%%%%%%%%%%%%%%%%%%%%%%%%%%%%%%%%%%%%%%%%%%%%%%%%%%%%%%%%%%%%%%%%%%%
%%%%%%%%%%%%%%%%%%%%%%%%%%%%%%%%%%%%%%%%%%%%%%%%%%%%%%%%%%%%%%%%%%%%%%%%%%%%%%%%
%%%%%%%%%%%%%%%%%%%%%%%%%%%%%%%%%%%%%%%%%%%%%%%%%%%%%%%%%%%%%%%%%%%%%%%%%%%%%%%%
\section{Retargeting}
\begin{frame}
   \vfill
   \begin{center}
     \Large II. Retargeting
   \end{center}
   \vfill
\end{frame}

\begin{frame}
  \frametitle{Objective}
  \begin{columns}
    \column{.45\paperwidth}
    \begin{center}
      \includegraphics[width=.45\paperwidth]{assets/mocap}
      \par
      \small
      Motion Reproduction
    \end{center}
    \column{.45\paperwidth}

  \begin{itemize}       
  \item Adapt acquired data on one actor to another.
  \item Adapt the motion to the capabilities of the new actor (i.e.\
  robot in our case).
  \end{itemize}
  \end{columns}

\end{frame}


\begin{frame}
  \frametitle{Optimization Pipeline}

  \begin{center}
    \begin{tikzpicture}
      [node distance=.8cm,
        start chain=going below,]
      \node[punktchain, join] (pre) {%
        \alert{Pre-Processing}\\
        \footnotesize
        Laplacian Deformation Energy, Bone Length constraint
      };
      \node[punktchain, join] (post) {%
        \alert{Conversion}\\
        \footnotesize
        Convert marker position into articular trajectory (IK)};
      \node[punktchain, join] (full) {%
        \alert{Full Problem}\\
        \footnotesize
        Laplacian Deformation Energy, all constraints
      };

      \draw[tuborg, decoration={brace}] let \p1=(pre.north), \p2=(pre.south) in
      ($(3, \y1)$) -- ($(3, \y2)$) node[tubnode,text width=4cm] {%
        Marker Position Optimization
      };
      \draw[tuborg, decoration={brace}] let \p1=(full.north), \p2=(full.south) in
      ($(3, \y1)$) -- ($(3, \y2)$) node[tubnode,text width=3cm] {%
        Joint Position Optimization
      };
    \end{tikzpicture}
  \end{center}
\end{frame}

\begin{frame}
  \frametitle{Pre-Processing: Marker Position Optimization}

  \begin{columns}
    \column{.45\paperwidth}
    \begin{center}
      \includegraphics[width=.22\paperwidth]{%
        figure/before.png}
      \par
      \small
      Marker Optimization.
    \end{center}
    \column{.45\paperwidth}

    Ensuring correct segment length while changing the motion as
    little as possible.

    \begin{description}
      \item[Cost] Laplacian Deformation Energy
      \item[Constraint] Bone Length
    \end{description}
  \end{columns}
\end{frame}

\begin{frame}
  \frametitle{Cost: Laplacian Deformation Energy}

  \alert{Laplacian Deformation Energy} quantify the interaction mesh
  deformation.

  \begin{equation}
    E_L(\mathbf{m}) = \sum_i \frac{1}{2} \| \mathbf{\delta}_i - L(\mathbf{m}_i) \|^2
  \end{equation}

  \begin{equation}
    L(\mathbf{m}_i) = \mathbf{m}_i - \sum_j w_{(i,j)} \mathbf{m}_j
  \end{equation}

  $m_i$ the $i$-th marker Euclidian position.\\
  $\mathbf{\delta}_i$ the original Laplacian Coordinate of the $i$-th marker.\\
  $w_{(i,j)}$ the weight of the $(i,j)$ link in the interaction mesh.
\end{frame}


\begin{frame}
  \frametitle{Full Optimization Problem Definition}

  \begin{description}
  \item[Cost] Laplacian Deformation Energy
  \item[Constraints] ~\\
    \begin{itemize}
    \item Joint Position, Velocity
    \item Robot stability (ZMP)
    \end{itemize}
  \end{description}
\end{frame}


\begin{frame}
  \frametitle{Full Pipeline Description}

  \begin{columns}
    \column{.45\paperwidth}
    \begin{center}
      \includegraphics[width=.44\paperwidth]{%
        figure/retargeting-preprocessing.jpg}
      \par
      \small
      Before (pink) and After (grey) optimization.
    \end{center}

    \column{.45\paperwidth}

    \begin{itemize}
    \item First, adapt the motion to the robot morphology.
    \item Second, make sure that the motion is feasible.
    \end{itemize}
    
  \end{columns}
\end{frame}


% Questions
\maxFrameImage{assets/hrp2-thx.jpg}


%%%%%%%%%%%%%%%%%%%%%%%%%%%%%%%%%%%%%%%%%%%%%%%%%%%%%%%%%%%%%%%%%%%%%%%%%%%%%%%%
%%%%%%%%%%%%%%%%%%%%%%%%%%%%%%%%%%%%%%%%%%%%%%%%%%%%%%%%%%%%%%%%%%%%%%%%%%%%%%%%
%%%%%%%%%%%%%%%%%%%%%%%%%%%%%%%%%%%%%%%%%%%%%%%%%%%%%%%%%%%%%%%%%%%%%%%%%%%%%%%%
\section{RobOptim Tutorial}
\begin{frame}
   \vfill
   \begin{center}
     \Large III. RobOptim Tutorial
   \end{center}
   \vfill
\end{frame}

\begin{frame}
  \frametitle{Tutorial}

  Time to work!

  \begin{itemize}
  \item You need: a laptopt with Ubuntu or Mac OS X, an installation of RobOptim.
  \item Tutorial is available online: \url{http://www.github.com/roboptim/roboptim-tutorial}
  \item No internet is available: a copy of the tutorial is on the USB key!
  \item 5 exercises, one hour and half: can be very long or very short
    depending on how much you are used to develop on your platform.
  \end{itemize}
\end{frame}

\end{document}
